\documentclass{article}
\usepackage[utf8]{inputenc}
\usepackage[brazil]{babel}
\usepackage{fancyvrb}
\usepackage[alf]{abntex2cite}

\title{
  Participa.br: Uma plataforma de participação democrática do governo Brasileiro
}

\author{
  Joenio Costa\\
  \texttt{joenio@joenio.me}
  \and
  Mariel Zasso\\
  \texttt{mariel.zasso@gmail.com}
  \and
  Paulo Meirelles\\
  \texttt{paulormm@unb.br}
  \and
  Ricardo Poppi\\
  \texttt{ricabras@gmail.com}
}

\begin{document}

\maketitle

\section{Introdução}

% * Motivação (contexto da participação no governo Brasileiro antes do Participa.br)

A participação social no Brasil representa princípio jurídico-institucional
presente na Constituição Federal de 1988, que a definiu como forma de afirmação
da democracia e da consolidação da cidadania. 

Ao incorporar esse princípio como referência para a gestão pública, o Governo
Federal aprimora os processos de interação do Estado com a Sociedade e cria as
condições institucionais para a prática da democracia participativa. 

Com isso, verifica-se que, além da crescente participação social nas decisões
governamentais, as políticas públicas ganham maior legitimidade, uma vez que
expressam as atuais condições socioeconômicas e culturais da população
brasileira em suas diferentes realidades regionais.

Na estrutura administrativa do Poder Executivo Federal, cabe à Secretaria-Geral
da Presidência da República (SG/PR) a função de intermediar as relações do
Governo com as entidades da sociedade civil, conforme competências definidas
pela Lei 10.683/2003 e pelo Decreto no. 7.688/2012. Assim, a SG/PR é órgão
incumbido de assessorar diretamente a Presidenta da República e os órgãos e
entidades do Governo Federal no relacionamento e na articulação com os
movimentos sociais, o que inclui a criação e a implementação de canais que
assegurem a consulta e a participação popular na discussão e na definição da
agenda prioritária do país.

O Brasil tem um rico histórico de efetivação da democracia participativa, sendo
reconhecido mundialmente. Os instrumentos institucionalizados como conselhos de
políticas públicas e conferências nacionais foram profundamente ampliados na
última década, contando com um legado volumoso de práticas e realizações. 

A maioria dos programas de governo já conta com participação social prevista em
pelo menos uma de suas etapas. As práticas trazidas pelas novas mídias e pela
cultura digital podem interagir nesses espaços fortalecendo, ampliando e
aprofundando a democracia participativa, especialmente neste novo século,
quando consideramos o contexto das redes sociais digitais.

%Desde outubro de 2011, a partir de uma oficina no seminário nacional de
%participação social, a Plataforma Federal de Participação Social (participa.br)
%começou a ser concebida conceitualmente.  No começo de 2013 uma versão teste da
%plataforma foi iniciada, e foi considerada a necessidade de envolver todos os
%cidadãos (especialmente aqueles que nas manifestações de junho de 2013
%demonstraram não se sentir representados pelo atual sistema político e que
%utilizaram as rede socias para se mobilizar e expressar o desejo de diálogo,
%incidência e representação sobre as decisões do poder público e suas políticas
%públicas). O desenvolvimento da Plataforma Federal de Participação Social, a
%partir de junho de 2013, obteve a máxima centralidade política para buscar
%novas formas de diálogo com a Sociedade.
%
%Incentivando os atores a conectar perfis, blogs e demais instâncias de produção
%de conteúdo na rede, o Portal de Participação Social (participa.br) se
%estabeleceu como espaço de participação social na rede e lançou as bases para
%se constituir como um repositório integrador do conhecimento sobre participação
%social, antes bastante disperso na internet e nas instâncias governamentais. 

(pendente\ldots)

\section{Plataformas digitais de participação}

% * Fundamentação (referências sobre governo aberto, ferramentas de participação digital, noosfero, etc)

A nossa sociedade passa por grandes transformações, especialmente na maneira
como lida com a informação disponível. O avanço da cobertura da Internet e das
redes sociais e a apropriação desses novos meios de comunicação pelas novas
gerações transformou as práticas comunicacionais e afetou diversos setores da
sociedade, inclusive a Política e os Governos. Essa profunda transformação
permitiu que a a informação fluísse de maneria rápida e direta para além dos
meios e intermediários tradicionais, dependendo cada vez menos dos meios de
comunicação de massa. 

Nos últimos anos, observou-se a popularização de ambientes e ferramentas que
digitalizam e amplificam as redes de colaboração e amizade, e permitem
publicação fácil e instantânea, a exemplo dos assim chamados blogs e serviços
de redes sociais. Esses ambientes, com grande adesão no Brasil, permitem a
mobilização e a rápida disseminação de conteúdos,  e possuem um grande alcance
pois trafegam por extensas redes de contatos. Percebendo as oportunidades desse
novo canal de interação, diversas empresas buscam acompanhar o comportamento
desses novos e potenciais clientes para estabelecer contato direto com esses
novos atores sociais nas redes.

A rapidez desse novo canal de interação permite que as solicitações e
reclamações postadas via redes sociais (e os diálogos subsequentes) sejam muito
mais rápidas do que as enviadas pelo antigos canais de interação, como correio,
por exemplo. O acompanhamento e sistematização de todas as demandas dos
clientes também ficou muito mais fácil e rápido. Com menos agilidade para se
apropriar e acompanhar essas mudanças, os Governos também tem feito algumas
incursões nesse novo meio, como criando canais oficiais para novas demandas,
propondo consultas públicas em metodologias abertas ou até propondo a
construção de uma constituição pela rede, com no caso recente da Islândia.

Desta forma, antes da existência do Portal Federal de Participação Social, a
Secretaria Geral da Presidência da República utilizava prioritariamente a
interação presencial, com o uso mínimo de interação via Internet para dialogar
com a Sociedade ou realizar debates e ações que envolvessem os mecanismos
formais de participação social instituídos pelo Decreto Presidencial nº
8.243/2014 (Decreto que institui a Política Nacional de Participação Social -
PNPS).

\section{O Participa.br}

(pendente\ldots)

% * Apresentar e descrever a plataforma Participa.br

O Participa.br é o Portal Federal de Participação Social, mantido pela SGPR
(Secretaria Geral da Presidência da República) com especial atenção de setores
de importância estratégica, como a SNAS/SGPR (Secretaria Nacional de
Articulação Social) e o Departamento de Participação Social (DPS). Trata-se de
um ambiente virtual de participação social, ou seja, um mecanismo de interação
social que utiliza tecnologias de informação e de comunicação, em especial a
internet, para promover o diálogo entre a administração pública federal e a
sociedade civil. O portal já acolheu diversos processos participativos desde o
seu lançamento, facilitando e ampliando o dialogo entre o Governo (Gestores
Públicos) e Sociedade (Cidadãos). 

O portal busca dar evidência às formas de participação existentes e previstas
no decreto da PNPS no sentido de contextualizar, organizar e facilitar o acesso
do cidadão às formas de incidir nas diversas etapas das políticas públicas do
governo brasileiro que estão sendo debatidas em suas comunidades.

O Portal dispõe de ambientes interativos e participativos para consultas
públicas e etapas virtuais de conferência, transmissão interativa de eventos e
reuniões, tornando-se um repositório transparente e agregador do conhecimento
sobre participação social disperso na rede. O portal de participação social se
propõe como um espaço onde a sociedade dialoga com diferentes agentes de
governo, trazendo contribuições e colaboração nas diversas etapas das políticas
públicas.

A plataforma organiza seus debates em torno de comunidades temáticas criadas a
partir do interesse da sociedade ou governo. A gestão das comunidades é
conjunta. A construção de um processo participativo dentro de uma comunidade
ocorre no objeto “trilha” a qual permite estabelecer um caminho colaborativo de
participação com diversas etapas. Cada etapa está ligada a uma ferramenta
digital de participação.  

Figura 4 - Diagrama de concepção de funcionamento do participa.br: comunidade
possui trilhas; trilha possui etapas; etapa possui ferramenta.

Cada comunidade criada pode usufruir de uma ou mais trilhas de participação.
Cada trilha pode ter uma ou mais etapas participativas, presenciais ou online.
Cada etapa participativa possui uma ferramenta principal empregada no processo
que está sendo desenvolvido naquela etapa.

Atualmente o Participa.br dispõe de mais de 24 comunidades temáticas. Cada
comunidade está construindo o seu caminho (trilha) e estabelecendo os seus
processos participativos (etapas e ferramentas). Algumas já possuem mais de uma
trilha de participação em funcionamento. As Comunidades são ambientes que
reúnem pessoas com interesses comuns. Facilitam a troca de ideias e a
interação, organizando os debates e aumentando o engajamento. 

Dentre estas, há comunidades temáticas que se constituem como espaços para a
proposição e o debate de ideias que poderão se tornar políticas públicas. As
comunidades temáticas devem ser geridas por representantes do Governo e da
Sociedade Civil. Os  órgãos de governo presentes, que participam de comunidades
temáticas, são: o  Ministério do Planejamento, Orçamento e Gestão (MPOG);
Ministério das Comunicações (MC); Universidade de Brasília (UnB); Serviço
Federal de Processamento de Dados (Serpro); Ministério da Justiça;
Secretaria-Geral da Presidência da República e Controladoria-Geral da União.
Algumas organizações da Sociedade Civil e Governos Estaduais também estão
presentes na plataforma: Associação Software Livre e Governo do Distrito
Federal.

O participa.br está descrito como um dos compromissos assumidos pelo Brasil no
Open Government Partnership (OGP) - Plano de Ação Brasileiro para a Parceria de
Governo Aberto.

\section{A construção do Participa.br}

(pendente\ldots)

% * Planejamento do participa (reuniões iniciais, pessoas envolvidas na
%   concepção, planejamento, primeiros passos)
%
% * Relato da construção/implementação (ferramentas, software livre, noosfero,
%   consultores, produtos desenvolvidos, instituições envolvidas: Presidência,
%   UnB, etc, Ministérios, …, pessoas)
%
% * Casos de uso relevantes do Participa.br (plano de participação, arena net
%   mundial, etc…)


\section{Contribuições laterais do Participa.br}

(pendente\ldots)

% * Legados (plataformas e ferramentas de software desenvolvidas, impacto em
%   outras iniciativas, projetos que nasceram inspirados no participa.br, pessoas
%   e grupos que tiveram formação em processos de participação por conta do
%   envolvimento no projeto, etc…)

\section{Conclusão}

(pendente\ldots)

% * O participa.br hoje e o futuro…

\bibliography{bibliografia}
\appendix

\end{document}
