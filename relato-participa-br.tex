\documentclass{article}
\usepackage[utf8]{inputenc}
\usepackage[brazil]{babel}
\usepackage{fancyvrb}
\usepackage[alf]{abntex2cite}

\title{
  Participa.br: Uma plataforma de participação democrática do governo Brasileiro
}

\author{
  Joenio Costa\\
  \texttt{joenio@joenio.me}
  \and
  Mariel Zasso\\
  \texttt{mariel.zasso@gmail.com}
  \and
  Paulo Meirelles\\
  \texttt{paulormm@unb.br}
  \and
  Ricardo Poppi\\
  \texttt{ricabras@gmail.com}
}

\begin{document}

\maketitle

\section{Introdução}

% (1) Motivação (contexto da participação no governo Brasileiro antes do Participa.br)

(pendente\ldots)

\section{Plataformas digitais de participação}

(pendente\ldots)

% (2) Fundamentação (referências sobre governo aberto, ferramentas de participação digital, etc)

\section{O Participa.br}

(pendente\ldots)

% (3) Planejamento do participa (reuniões iniciais, pessoas envolvidas na
%     concepção, planejamento, primeiros passos)
%
% (4) Relato da construção/implementação (ferramentas, software livre, noosfero,
%     consultores, produtos desenvolvidos, instituições envolvidas: Presidência,
%     UnB, etc, Ministérios, …, pessoas)
%
% (5) Casos de uso relevantes do Participa.br (plano de participação, arena net
%     mundial, etc…)

\section{Contribuições laterais do Participa.br}

(pendente\ldots)

% (6) Legados (plataformas e ferramentas de software desenvolvidas, impacto em
%     outras iniciativas, projetos que nasceram inspirados no participa.br, pessoas
%     e grupos que tiveram formação em processos de participação por conta do
%     envolvimento no projeto, etc…)

\section{Conclusão}

(pendente\ldots)

% (7) O participa.br hoje e o futuro…

\bibliography{bibliografia}
\appendix

\end{document}
